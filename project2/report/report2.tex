\documentclass[article,a4paper,oneside]{article}
%% Math stuffs
\usepackage[utf8]{inputenc}
\usepackage[english]{babel}
\usepackage[T1]{fontenc}
\usepackage{amssymb}
\usepackage{amsmath}
\usepackage{amsthm}
\usepackage{todonotes}
\usepackage{hyperref}
\usepackage{graphicx}
\newtheorem{thm}{Theorem}
\newtheorem{definition}{Definition}
%%We have quite a bit of math inline, so let's remove the paragraph indent and instead move it a bit down
\parindent 0pt
\parskip 4mm
%% for easy Matrix notation
\newcommand{\+}[1]{\ensuremath{\boldsymbol{#1}}}

%%For syntaxhighligting when needed:
%%remember to invoke pdflatex with -shell-escape when not commented out
\usepackage{minted}

\begin{document}
\title{
Randomized Algorithms Project 2\\
A randomized fingerprinting algorithm for efficient computation of multiset equality.
}

\author{
  Mads Ravn - 20071580\\
  Bo Mortensen - 20073241\\
  Johan Abildskov - 20063623
}

\date{\today}

\maketitle

\newpage

\subsection*{Randomized fingerprinting of multisets}
\subsubsection*{Introduction}
In the following report we will describe the results from implementing a fingerprinting algorithm for determining multiset equality based upon $2$-$Universal$ hash functions and \emph{Schwartz-Zippel}.
The issue of determining multiset equality with a deterministic approach is that there are no obvious way that is better than using an efficient sorting algorithm and test for equality. Using the \emph{Schwartz-Zippel} approach also allows us to trivially extend the algorithm to a \emph{streaming} implementation, working efficiently on large datasets. No such approach seems to be possible for the deterministic approach.

\subsection*{Equality of hashed multisets}
\begin{definition}{2-Universal hash functions}\\
A hash function h(x) from some Universe \emph{U} to a table {T}, where $|U| >> |T|$, is 2\emph{-}universal iff $x \neq y, x,y \in U \Rightarrow Pr[h(x) = h(y)] \leq \frac{1}{|T|}$
\end{definition}
In the rest of the report $T$ will be $\mathbb{F}_p$ where $p = 2^{31} - 1$, which is prime.
We now consider two distinct multisets $X = \left[x_1,\cdots, x_n\right]$ and $Y = \left[y_1, \cdots, y_n\right]$. We note that we can assume that both sets have the same size, as it is trivial to determine inequality, if the multisets have the different sizes.
We wish to show that given a hash function $x$ selected uniformly randomly from the family of 2-universal hash functions $H$, $h(X) = \left[h(x_1),\cdots, h(x_n)\right]$ is equal to $h(Y) = \left[h(y_1),\cdots,h(y_n)\right]$ with at most probability $\frac{n}{p}$.
\subsection*{A 2-universal family of hash functions}
As we would like to do the arithmetic modulo $p = 2^{31}-1$, we wish to create a family of hash functions that allows us to do work in $\mathbb{F}_p$, even though our elements are of size up to $2^{80}$.

We define $H_s=\{\ h_{a_1,\ldots,a_s}\ | \ a_1,\ldots,a)s\in \mathbb{F_p}\ \}$,
where $h_{a_,\ldots,a_s}(x_1,\ldots,x_s)$.

In order to show that $H_s: 2^{80} \rightarrow 2^p$ indeed is 2-universal, we have to prove that the probability of a collision $h(x) = h(y)$ is less than $\frac{1}{p}$ if $x \neq y$.

Given two elements $x,y \in U, x \neq y$ and a hash function $h_s \in H_s$, the following holds:
\begin{align*}
h(x) &= h(y) \iff \\
\sum_{i=1}^{s}a_ix_i &= \sum_{i=1}^sa_iy_i \iff \\
\sum_{i=1}^{s}a_ix_i - \sum_{i=1}^sa_iy_i &= 0 \iff \\
a_1x_1 - a_1y_1 + \sum_{i=2}^{s}a_ix_i - \sum_{i=2}^sa_iy_i &= 0 \iff \\
\sum_{i=2}^{s}a_ix_i - \sum_{i=1}^sa_iy_i &= a_1(y_i-x_1) \iff \\
\frac{\sum_{i=2}^{s}a_ix_i - \sum_{i=1}^sa_iy_i}{y_1-x_1} &= a_1 \\
\end{align*}
\end{document}
