\documentclass{report}
\usepackage{amsmath}
\usepackage{amssymb}
\usepackage{amsthm}
\newtheorem{thm}{Theorem}
\newcommand{\+}[1]{\ensuremath{\boldsymbol{#1}}}
\begin{document}
\chapter{April 10th}
\section{Fingerprinting}
The motivation is that we might be able to determine whether or not it is feasible to find a concrete solution to a problem, that has a bad time complexity.
This might save a lot of time trying to solve an impossible solution.
\par
\subsection{Matrix multiplication identity}
One example is matrix operations. All non-trivial matrix operations has a complexity of $\mathcal{O}(n^3$. This is operations like determinant, inverse and multiplication. Some have theoretical lower bounds, but are impractical or with high constants.
\par
A place where fingerprinting is useful is determining matrix multiplication identity. That is given matrices $A, B \text{ \& } C$ determine whether $AB = C$.
Obviously a solution is to calculate $\+A\+B$ and compare with $\+C$, this would though obviously be $\mathcal{O}(n^3)$.
\par
Another option is to select a random vector $r \in \left\{0,1\right\}^n$ and compute $\+A(\+Br)= \+Cr$. Each of the operations are of order $\mathcal{O}(n^2)$.
\subsubsection{Analysis}
\begin{thm}[Rajeev]\label{matrixfingerprinting}
Let $\+A,\ \+B\ and\ \+C$ be $n \times n$ matrices over $\mathbb{F}$ such that $\+A\+B \neq \+C$. Then for $\+{r}$ chosen uniformly at random from $\left\{0,1\right\}^n,$\\ $Pr[\+A\+B\+r = \+C\+r] \leq 1/2$.\\
\end{thm}
\begin{proof}
Let $\+D = \+A\+B-\+C$. As $\+A\+B \neq \+C$ we know that $\+D$ is not the all-zeroes matrix. We now wish to get an upper bound on $Pr[Dr = 0]$. Without loss of generality assume that the first row of $\+D$ has a non-zero entry, and that all non-zero entries of that row precedes any zero entries. 
Let this row be named $\+d$ and assume that the first $k > 0$ entries of $\+d$ is non-zero.
as the first entry of $\+D\+r$ is $\+d^T\+r$, we're going to focus on this.
\par
$\+d^T\+r =$ iff $r_1 = \frac{\sum_{i=2}^{k}d_ir_i}{d_1}$.
Using the Principle of Deferred Decisions we can assume that the last selected entry in $\+r$ will be $r_1$. This fixes the right hand side of the above equation at some value $v \in \mathbb{F}$. Since $r_1$ is uniformly distributed over a set of size 2, the probability that it is equal to $v$ cannot exceed 1/2.
\end{proof}

\subsection{Van der Monde Matrices}
Another use for Randomized finger printing is using calculating the determinant of a Vandermonde matrix to determine identity between multivariate polynomials. This might be useful in cases where we do not have the polynomials explicitly represented but rather implied.
The Vandermonde Matrix ca be used to figure this, but calculating this determinant is in many cases in feasible as it has a number of terms around $n!$ terms.
\par
The Vandermonde matrix is defined in terms of the variables: $x_1 \ldots x_n$ as follows:\\
$M = \begin{pmatrix}
  1 & x_1 & x_1^2 & \cdots & x_1^{n-1} \\
  1 & x_2 & x_2^2 & \cdots & x_2^{n-1}\\
  \vdots & \vdots & \vdots & \ddots & \vdots \\
  1 & x_n & x_n^2 & \cdots & x_n^{n-1}
\end{pmatrix}
$
The idea is that $|M| =\prod_i<j (x_j - x_i)$, which we would like to prove, but we don't know whether it holds true.
For this reason a strategy might be to verify this on a few large matrices before abandoning the approach.
\par
An issue with this is that we can end up with extremely large number in our intermediate calculations. We can solve this by doing our calculations for finite field $\mathbb{F}_p$. $|\mathbb{F}_p|$ needs to be larger than the degree of the polynomial since the polynomial can have that many roots.. Which is $\mathbb{O}(n^2)$. $p$ needs to be a prime, in order for us to guarantee that we don't introduce additional roots by keeping our computations in $\mathbb{F}_p$.
\end{document}
