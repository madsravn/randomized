\documentclass[article,a4paper,oneside]{memoir}
%% Math stuffs
\usepackage[utf8]{inputenc}
\usepackage[english]{babel}
\usepackage[T1]{fontenc}
\usepackage{amssymb}
\usepackage{amsmath}
\usepackage{amsthm}
\usepackage{todonotes}
\newtheorem{thm}{Theorem}

%%We have quite a bit of math inline, so let's remove the paragraph indent and instead move it a bit down
\parindent 0pt
\parskip 4mm
%% for easy Matrix notation
\newcommand{\+}[1]{\ensuremath{\boldsymbol{#1}}}

%%For syntaxhighligting when needed:
%%remember to invoke pdflatex with -shell-escape when not commented out
\usepackage{minted}

\begin{document}
\title{
Randomized Algorithms Project 1\\
Analysis and experiments with randomized kth find and randomized quicksort
}

\author{
  Mads Ravn - 20071580\\
  Bo ``Metal-Bo'' Mortensen - 20073241\\
  Johan Abildskov - 20063623
}

\date{\today}

\maketitle

\newpage

\tableofcontents*


\chapter{Introduction}
In the following report we will prove a bound on finding the $k$th smallest as well as show experiments relating to the theoretical bounds. We will do a similar experimentation for \emph{randomized QuickSort}.

\chapter{\texttt{FIND}}
In this section we look at the the problem of finding the $k$th smallest integer in a list $L$. That is, given an unsorted list $L$, return the $k$th smallest element.
The algorithm is given here:
\begin{center}\begin{minipage}{5in}
\underline{Algorithm: FIND}\\
\begin{tabular}{ll}
Input: & $L=[a_1,\ldots,a_n]$, a nonempty list of distinct numbers\\
& $k$, an integer such that $1\leq k\leq n$.\\ 
Output: & $b$, where $b\in L$ and $|\{a\in L:a\leq b\}|=k$\\
Method:\\
{\hfill}1. & Select $e$ randomly from $L$ using the uniform distribution\\
{\hfill}2. & Split $L'=L-\{e\}$ into the two sublists\\
& \ \ $L_1=[a_i\in L|a_i<e]$\\
& and\\
& \ \ $L_2=[a_i\in L|a_i>e]$\\
& by comparing $e$ to each element of $L'$.\\
{\hfill}3. & If $|L_1|=k-1$ then return $e$.\\
& If $|L_1|>k-1$ then make a recursive call on $L_1$ and $k$.\\
& \begin{minipage}{4in}
If $|L_1|<k-1$ then make a recursive call on $L_2$ and $k-1-|L_1|$.
\end{minipage}
\end{tabular}
\end{minipage}\end{center}
As we can see the, approach is quite similar to that of a randomized quicksort.
We are following the suggested approach from the project description.
\par

Let $\pi$ be the unique permutation that sorts the elements of $L$,
i.e. $a_{\pi(1)}<\cdots <a_{\pi(n)}$.  Define the indicator variables

$$Z_{i}=\left\{\begin{array}{ll}
1, & \mbox{
\begin{minipage}[t]{4in}
  if $a_{\pi(i)}$ is selected as the random element $e$ in line 1
  during any of the recursive executions of FIND
\end{minipage}
}
\\
0, & \mbox{ otherwise}
\end{array}\right.$$

Define:

$$Z=\sum_{i\neq k} Z_i$$

\subsection{Subproblem 1.1}
First, we're going to argue that $1+Z$ denotes the recursion depth during an execution of \texttt{FIND}. Each index in $Z_i$ that is set to $1$ as defined above represents an added layer in the recursion, as this represents that the element at some point was chosen as the pivot element around which we create the two lists, one of which we will recurse in. The additional $1$ comes from the fact that we have defined $Z$ as the sum of indicator variables, except for $i=k$. This element itself obviously represents the point at which we have chosen the $k$th element as our pivot, and this adds the last layer of our recursion.
\par

\subsection{Subproblem 1.2}
In order to show a bound on the time complexity of \texttt{FIND}, we need to look at how large $Z$ becomes. So we will show that $$Pr[Z_{i}]=\frac{1}{|k-i|+1}$$.
\par
Consider that permutation $\pi$ as described above. We will look at the probability that any $a_{\pi(i)}$ is selected in the execution of \texttt{FIND}. Without loss of generality we look only at $\pi(i) < \pi(k)$. Consider now the subset of elements $$S = \lbrace a_{\pi(i)},\cdots,a_{\pi(k)}\rbrace$$ it is seen that the element $a_{\pi(i)}$ will be chosen as a pivot element \emph{iff} it is the \emph{first} element of $S$ to be chosen. If another element is chosen from $S$ then $a_{\pi(i)}$ will end up in the list that the algorithm does not do recursive work in. As each element in $S$ is equally likely to be chosen and the size of $S$ is $|k - i| + 1$, the probability that $i$ will be chosen first is $\frac{1}{|k -1|}$. We have thus proved that $$Pr[Z_{i} = 1] = \frac{1}{|k-i|+1}$$

\subsection{Subproblem 1.3} 
Prove that $E[Z]\leq \mu = 2\ln n$.\\
For any $k, 0<k<n$ we have that $Z = \sum_{i=1}^{n} \frac{1}{|k-i|+1}$.
This gives cause to the following: \todo{Lad os lige sikre os det er helt korrekt matematisk set (større-end og lighed). leq 2Hn er fordi det er kte og ikke nte harmoniske tal}

\begin{align*}
E[Z] &= \sum_{i=1}^{n} \frac{1}{|k-i|+1}
\\ &\leq \sum_{i=1}^{k-1} \frac{1}{i} + \sum_{i=k+1}^{n} \frac{1}{i}
\\ &\leq \sum_{i=1}^{k-1} \frac{1}{i} + \sum_{i=1}^{k-1} \frac{1}{i}
\\ &= 2\sum_{i=1}^{k} \frac{1}{i+1}
\\ &\leq 2H_n
\\ &\approx 2lnn \qedhere
\end{align*}

\subsection{Subproblem 1.4}
Argue that $Z_1, Z_2, \ldots, Z_{k-1}$ are independent (and, similarly, $Z_{k+1}, Z_{k+2}, \cdots, Z_n$ are independent).
\par
To show that $Z_i$ and $Z_j$ are independent for $i,j < k$, consider the case $Z_i = 1$. This means that in the interval $\mathbb{I} = [Z_i,..,Z_k]$, $Z_i$ was the first to be chosen as the random pivot element.
For any $Z_j, j \neq i$ we're considering the interval $\mathbb{J} = [Z_j,\ldots, Z_k]$, as $i \neq j \Rightarrow \mathbb{I} \neq \mathbb{J}$, we invoke the \emph{Principle of deferred decisions} and see that the two intervals are not related, and thus the choice of $Z_i$ and $Z_j$ is independent.
\subsection{Subproblem 1.5}
Prove that $\Pr (Z > 2\mu)\leq 2n^{-c}$.
\\
From (1.4) we have that $Z_1, Z_2, \ldots, Z_{k-1}$ are independent and $Z_{k+1}, Z_{k+2}, \ldots, Z_n$ are independent, and as such, the Chernoff bound technique can be applied in this situation. We denote the set of $Z_1, Z_2, \ldots, Z_{k-1}$ as $A$ and the set of $Z_{k+1}, Z_{k+2}, \ldots, Z_n$ as $B$. We know from (1.3) that the expected value of $A$ and $B$ are upper bounded by $\ln n$. We now utilize the Chernoff bound technique to this $A$.

$$\Pr \left( X > (1+\delta)\mu \right) < \left[ \frac{e^\delta}{(1+\delta)^{1+\delta}} \right]^\mu$$

\begin{align*}
  Pr( A > 2\mu) & <  \left[ \frac{e^1}{(1+1)^{1+1}} \right] ^{\ln n} \\
  &= \frac{e^{\ln n}}{4^{\ln n}}\\ 
  &= \frac {n}{e^{\ln(4)\ln(n)}} \\
  &= \frac{n}{n^{\ln(4)}}\\
  &= n^{1-\ln (4)}\\
\end{align*}

\todo{Skal vi have en mere beskrevet overgang fra udregningen til at den grænse gælder? Hvad med det 1 vi smider væk ved at dele Z op i A og B?}
The exact same thing applies for $B$ and now we have $\Pr ( Z > 2\mu)$ expressed as $\Pr ( A > 2\mu) + \Pr( B > 2\mu)$. We now have that $\Pr ( Z > 2\mu) \leq 2n^{1-\ln(4)}$ and thusly have a constant $c=\ln(4)-1$.


\subsection{Subproblem 1.6}

\todo{For each n, k pair you should run a large number of experiments (at least, pref more).}
\todo{Når vi når hertil, så formater lige opgavestillingen lidt bedre}

\chapter{\texttt{QUICKSORT}}

\subsection{Subproblem 1.7}
$E[X] \leq \mu_s = 2n\ln n$
\\\\
$\Pr (X > 3\mu_s) \leq 2n^{-d}$

$n\cdot \Pr[Z>3\mu]\leq n\cdot \left [ \frac{e^2}{3^3} \right ]^{2\ln n}\\
=n\cdot\frac{e^{4\ln n}}{27^{2\ln n}}\\
=\frac{n^5}{27^{2\ln n}}\\
=\frac{n^5}{e^{\ln 27 \ln n^2}}\\
=\frac{n^5}{n^{2\ln 27}}\\
=\frac{n^5}{n^{\ln 27}n^{\ln 27}}\\
=n^{5-2\ln 27}$


$\Pr[X>3\mu_s]\leq \left [ \frac{e^2}{3^3} \right ]^{2n\ln n}\\
=\frac{e^{4n\ln n}}{27^{2n\ln n}}\\
=\frac{n^{4n}}{e^{\ln 27\cdot 2n\ln n}}\\
=\frac{n^{4n}}{n^{\ln 27\cdot 2n}}\\
=n^{4n-\ln(27)\cdot 2n}\\
=n^{n(4-2\ln 27)}\\
=n^{n(4-\ln 27^2)}$
\subsection{Subproblem 1.8}

\subsection{Implementation and Experiments}
\todo{Write something about this}
\end{document}
